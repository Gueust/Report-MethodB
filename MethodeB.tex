\documentclass[10pt,a4paper]{article}
% Libraries
\usepackage[utf8]{inputenc} % [utf8] for linux, latin1 for windows
\usepackage[french]{babel} %\usepackage[english]{babel}
\usepackage{fullpage}
\usepackage{amsmath}
\usepackage{amsfonts}
\usepackage{amssymb}
\usepackage{graphicx} 
\usepackage{longtable}
\usepackage{algorithm}
\usepackage{listings}
\usepackage{algorithmic}
\usepackage{float}

\usepackage[center]{caption}
\usepackage{subcaption}

%\usepackage{calc}
\usepackage{amsthm}
%\usepackage[usenames,dvipsnames]{xcolor}
%\usepackage{tikz}
%\usepackage[all]{xy}

\usepackage[hidelinks]{hyperref}

\newtheorem{prop}{Proposition}[section]
\newtheorem{theorem}{Theorem}[section]
\newtheorem{lemma}{Lemma}[section]
\newtheorem{definition}{Definition}[section]
\newtheorem{requirement}{Requirement}[section]
\newtheorem{remark}{Remark}[section]
\newtheorem*{remarks}{Remarks}
\newtheorem{hypothesis}{Hypothesis}[section]
\newtheorem{example}{Example}

\setcounter{secnumdepth}{3}

\author{Jean-Baptiste Lespiau}
\title{System Optimal Dynamic Traffic Assignment with Partial Compliance}


%\renewcommand{\thechapter}{\Roman{chapter})}
%\renewcommand{\thesection}{\Roman{section}} %Alph
%\renewcommand{\thesubsection}{\arabic{subsection})}
%\renewcommand{\thesubsubsection}{\alph{subsubsection})}
%\renewcommand{\theparagraph}{\engrec{paragraph})}

\begin{document}

\begin{abstract}
Description du rapport
\end{abstract}

\iffalse
0) Courte intro: on va parler de quoi ?
Schema : Meteor: page 374
L'Atelier B inventé en 92 par Abrial, développé par ??.
Utilisé dans tels projets (Météor etc)
Une phrase qui décrit l'idée générale:
- une méthode pour spécifier, concevoir et implémenter des logiciels
- une théorie mathématique basée sur les ensembles et les relations
- un langage
- un ensemble d'outils autour de l'Atelier B
1)J-B Théorie des ensembles + théorie des relations (en annexe dans le rapport ?)
Exemples
3)[Charles] Explication des machines abstraites et de tous leurs champs
Définie dans Dossier-Technique page 9
Exemple de la bibliothèque: Cours B Part I 2007
4bis) Preuves
5)[Charles] Raffinement d'une machine abstraite (Dossier Technique page 10)
Rajouter un exemple (commence page 193 de Spécification avec B)
5bis) Obligation de preuves des raffinements
6) Preuves et obligations de preuves
7)[J-B] Implémentation d'un raffinement (définir ce qu'il doit contenir ou non).

A la fin:
Les tableaux pour donner les équivalences entre symboles mathématiques et grammaire.
\fi

\section{Introduction}

\section{Théorie sous-jacente de la méthode B}
\section{Description de machines abstraites}
\subsection{Preuve d'invariants}

\section{Raffinement}

\section{Implémentation de modèles}

\appendix

\section{Théorie ZFC}

\subsection{Z}
\subsection{F}
\subsection{Théorème du choix}

% ===================================
% Bib
\bibliographystyle{plain}
\bibliography{biblio.bib}
% ===================================

\end{document}