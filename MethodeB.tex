\documentclass[10pt,a4paper]{article}
% Libraries
\usepackage[utf8]{inputenc} % [utf8] for linux, latin1 for windows
\usepackage[french]{babel} %\usepackage[english]{babel}
\usepackage{fullpage}
\usepackage{amsmath}
\usepackage{amsfonts}
\usepackage{amssymb}
\usepackage{graphicx} 
\usepackage{longtable}
\usepackage{algorithm}
\usepackage{listings}
\usepackage{algorithmic}
\usepackage{float}

\usepackage[center]{caption}
\usepackage{subcaption}

%\usepackage{calc}
\usepackage{amsthm}
%\usepackage[usenames,dvipsnames]{xcolor}
%\usepackage{tikz}
%\usepackage[all]{xy}

\usepackage[hidelinks]{hyperref}

\newtheorem{prop}{Proposition}[section]
\newtheorem{theorem}{Theorem}[section]
\newtheorem{lemma}{Lemma}[section]
\newtheorem{definition}{Definition}[section]
\newtheorem{requirement}{Requirement}[section]
\newtheorem{remark}{Remark}[section]
\newtheorem*{remarks}{Remarks}
\newtheorem{hypothesis}{Hypothesis}[section]
\newtheorem{example}{Example}

\newtheorem{axiom}{Axiome}
\renewcommand{\theaxiom}{\Roman{axiom}} %\renewcommand{\theaxiom}{\thechapter.\thesection.\arabic{axiom}}

\setcounter{secnumdepth}{3}

\author{Jean-Baptiste Lespiau}
\title{System Optimal Dynamic Traffic Assignment with Partial Compliance}

\newcommand{\Bequal}{\mathrel{\widehat{=}}}


%\renewcommand{\thechapter}{\Roman{chapter})}
%\renewcommand{\thesection}{\Roman{section}} %Alph
%\renewcommand{\thesubsection}{\arabic{subsection})}
%\renewcommand{\thesubsubsection}{\alph{subsubsection})}
%\renewcommand{\theparagraph}{\engrec{paragraph})}

\begin{document}

\begin{abstract}
Description du rapport
\end{abstract}

\iffalse
0) Courte intro: on va parler de quoi ?
Schema : Meteor: page 374
L'Atelier B inventé en 92 par Abrial, développé par ??.
Utilisé dans tels projets (Météor etc)
Une phrase qui décrit l'idée générale:
- une méthode pour spécifier, concevoir et implémenter des logiciels
- une théorie mathématique basée sur les ensembles et les relations
- un langage
- un ensemble d'outils autour de l'Atelier B
1)J-B Théorie des ensembles + théorie des relations (en annexe dans le rapport ?)
Exemples
3)[Charles] Explication des machines abstraites et de tous leurs champs
Définie dans Dossier-Technique page 9
Exemple de la bibliothèque: Cours B Part I 2007
4bis) Preuves
5)[Charles] Raffinement d'une machine abstraite (Dossier Technique page 10)
Rajouter un exemple (commence page 193 de Spécification avec B)
5bis) Obligation de preuves des raffinements
6) Preuves et obligations de preuves
7)[J-B] Implémentation d'un raffinement (définir ce qu'il doit contenir ou non).

A la fin:
Les tableaux pour donner les équivalences entre symboles mathématiques et grammaire.
\fi

\section{Introduction}
Exemple de citation: 
\begin{itemize}
\item Météor \cite{behm1999meteor}
\item Spécification avec B  \cite{habrias2006specifications}
\end{itemize}

\section{Théorie sous-jacente de la méthode B}
\section{Description de machines abstraites}
% Forcer des espaces du plus petit au plus grand
% \, \; \: \quad \qquad
\textbf{MACHINE}

\qquad $registeredP \ \subseteq \ Passenger \wedge registeredB \subseteq Luggage \wedge$

\qquad $registeredP \subseteq Passenger \wedge registeredB \subseteq Luggage \wedge$

\qquad $f \Bequal f$

\textbf{SETS}
\indent
\indent
\indent
\textbf{CONSTANTS}

\subsection{Preuve d'invariants}

\section{Raffinement}

\section{Implémentation de modèles}

\appendix

\section{Théorie ZFC}

La théorie des ensembles de Zermelo est publiée en 1908 et instaure une axiomatique pour formuler une théorie moderne des ensembles qui n'est pas confronté aux paradoxes de la théorie de Cantor (par exemple, le paradoxe lié à l'existence d'un ensemble qui contiendrait tous les ensembles).

En 1922, Fraenkel et Skolem ajoutent des axiomes et le système d'axiomes en résultant est connu sous le nom de théorie de Zermelo-Fraenkel, abrégé en ZF.

On utilise la logique du premier ordre.

\subsection{Théorie de Zermelo (Z)}

\begin{axiom}[Axiome d'extensionnalité] Cet axiome définit l'égalité entre ensemble comme l'égalité de leurs éléments.
\begin{align}
\forall A\ \forall B \
\left[ 
\forall x \left( x \in A \iff x \in B  \right) \Rightarrow A = B
\right] 
\end{align}
\end{axiom}

\begin{axiom}[Axiome de la paire] Cet axiome définit, à partir deux ensemble A et B, un nouvel ensemble paire contenant A et B exactement.
\begin{align}
\forall A \ \forall B \ \exists C \forall x \left[ x \in C \iff \left( x = A \vee x = B \right) \right]
\end{align}
\end{axiom}

\begin{axiom}[Axiome de la réunion] Cet axiome définit, à partir d'un ensemble A, un nouvel ensemble $\bigcup A$ contenant les éléments des éléments de A. Avec l'axiome de la paire on peut définir l'union avec $A \cup B = \bigcup \left\{A, B\right\}$.
\begin{align}
\forall A\ \exists B\ \forall C\ \left( C\in B \iff \exists D\ \left( D\in A \wedge C\in D \right) \right)
\end{align}
\end{axiom}


\begin{axiom}[Schéma d'axiomes de compréhension] Aussi appelé, schéma d'axiomes de séparation, il permet de définir la construction d'ensembles en compréhension. Étant donné un ensemble $A$ et une propriété $\varphi$, il affirme l'existence de l'ensemble B des éléments de A vérifiant la propriété $\varphi$.

Ainsi, pour toute propriété $\varphi$ ne contenant pas d'autre variable libre que $x, a_1 \ldots a_p$ on définit l'axiome suivant :
\begin{align}
\forall a_1 \ldots a_n \ \forall A \ \exists B \ \forall x \ \left( x \in B \iff \left[ x \in A \wedge \varphi \left(x, a_1, \ldots, a_n \right) \right] \right) 
\end{align}
\end{axiom}

\begin{axiom}[Axiome de l'infini] Cet axiome définit qu'il existe un ensemble auquel appartient l'ensemble vide et qui est clos par application du successeur$ x \rightarrow x \cup {x}$. Il permet ainsi de construire un ensemble qui contient une représentation des entiers naturels. On note que l'ensemble vide existe comme conséquence de la logique du premier ordre du schéma d'axiomes de compréhension.
\begin{align}
\exists A \left( \emptyset \in A \land \forall x (x \in A \Rightarrow x \cup \{x\} \in A) \right) 
\end{align}
\end{axiom}

\begin{axiom}[Axiome de l'ensemble des parties] Cet axiome définit que pour tout ensemble $A$ il existe un ensemble auquel appartiennent exactement tous les sous-ensembles de $A$.
\begin{align}
\forall A \ \exists P \ \forall B \left[B \in P \Leftrightarrow \forall C \left( C \in B \Rightarrow C \in A \right) \right] 
\end{align}
\end{axiom}

\subsection{Théorie de Zermelo-Fraenkel (ZF)}

La théorie ZF ajoute deux axiomes à l'axiomatique déjà présentée.

\begin{axiom}[Axiome de fondation] Tout ensemble non vide x possède un élément y tel que x et y soit disjoint.
\begin{align}
\forall x [ \exists a ( a \in x) \Rightarrow \exists y ( y \in x \land \lnot \exists z (z \in y \land z \in x))]. 
\end{align}
\end{axiom}

\begin{axiom}[Schéma d'axiomes de remplacement] Ce schéma étend le schéma d'axiomes de compréhension de la théorie de Zermelo. Il définit que un ensemble A étant donné, son image par une relation fonctionnelle est un ensemble.

Ainsi, pour toute propriété $\phi$ ne contenant pas d'autre variable libre que $x, y, a_1 \ldots a_p$ on définit l'axiome suivant :
\begin{align}
\forall A \ \forall a_1 \forall \ a_2 \ldots \forall a_n \ 
\bigl[ \forall x ( x\in A \Rightarrow \exists! y\,\phi ) \Rightarrow \exists B \ \forall x \bigl(x\in A \Rightarrow \exists y (y\in B \land \phi) \bigr) \bigr]
\end{align}
\end{axiom}

\subsection{Théorie de Zermelo-Fraenkel avec axiome du choix (ZFC)}

\begin{axiom}[Axiome du choix] Étant donné un ensemble X d'ensembles non vides, il existe une fonction définie sur X, appelée fonction de choix, qui à chacun d'entre eux associe un de ses éléments.
\begin{align}
\forall X \left[ \emptyset \notin X \Rightarrow \exists f: X \rightarrow \bigcup X, \quad \forall A \in X \, ( f(A) \in A ) \right] \,. 
\end{align}
\end{axiom}

% ===================================
% Bib
\bibliographystyle{plain}
\bibliography{biblio.bib}
% ===================================

\end{document}